\documentclass[12pt]{cdblatex}
\usepackage{examples}
\usepackage{fancyhdr}
\usepackage{footer}

\begin{document}

% ============================================================================================
\bgroup
\CdbSetup{action=hide}
\begin{cadabra}
   import cdblib
   checkpoint_file = 'tests/semantic/output/example-10.json'
   cdblib.create (checkpoint_file)
   checkpoint = []
\end{cadabra}
\egroup

\clearpage

% ============================================================================================
\section*{Example 10 The determinant of the metric}

Our game here is to compute (the leading terms) in $\det g$ of the metric in RNC form
\begin{dgroup*}
   \Dmath*{g_{a b}(x) = \cdb{ex-10.gab.000}+\cdots}
\end{dgroup*}
For the sake of simplicity let's assume that we are working in 3-dimensions. The following
analysis is easily generalsied to other dimensions (and the final answers for $\det g$ and
friends are unchanged).

Define $\eps^{abc}_{ijk}$ by
\begin{align}
   \eps^{abc}_{ijk} =
        \delta^a_i \delta^b_j \delta^c_k - \delta^b_i \delta^a_j \delta^c_k
      + \delta^c_i \delta^a_j \delta^b_k - \delta^c_i \delta^b_j \delta^a_k
      + \delta^b_i \delta^c_j \delta^a_k - \delta^a_i \delta^c_j \delta^b_k
\end{align}
It is easy to see that $\eps^{abc}_{ijk}$ is anti-symmetric in both its upper and lower
indices. A trivial computation shows that for any $3{}\times{}3$ square matrix $M_{ab}$,
\begin{align}
   \eps^{abc}_{123} M_{1a} M_{2b} M_{3c}
   = \left(
          \delta^a_1 \delta^b_2 \delta^c_3 - \delta^b_1 \delta^a_2 \delta^c_3
        + \delta^c_1 \delta^a_2 \delta^b_3 - \delta^c_1 \delta^b_2 \delta^a_3
        + \delta^b_1 \delta^c_2 \delta^a_3 - \delta^a_1 \delta^c_2 \delta^b_3
     \right)M_{1a} M_{2b} M_{3c}
   = \det M
\end{align}
This can be easily generalised to
\begin{align}
   \eps^{abc}_{ijk} M_{pa} M_{qb} M_{rc}
   =
   \begin{cases}
      \pm \det M &\text{when $(ijk)$ and $(pqr)$ are permutations of $(123)$}\\
      0 & \text{otherwise}
   \end{cases}
\end{align}
The $\pm$ sign in the above depends on the particular permutations of $(ijk)$ and $(pqr)$. If
both permutations are even or both odd then the sign is $+1$ otherwise the sign is $-1$.
The same arguments can also be applied to a matrix inverse $N^{-1}$ leading to
\begin{align}
   \eps^{ijk}_{uvw} N^{pu} N^{qv} M^{rw}
   =
   \begin{cases}
      \pm \det {N^{-1}} &\text{when $(ijk)$ and $(pqr)$ are permutations of $(123)$}\\
      0 & \text{otherwise}
   \end{cases}
\end{align}
Note that the $\pm$ in this case will match exactly that for the case of $\det M$. Thus,
multiplying both expressions and summing over all choices for $(ijk)$ and $(pqr)$ leads
to
\begin{align}
   \sum_{\substack{(ijk)\\(pqr)}}\left(\det N^{-1}\right) \det M
   = \eps^{ijk}_{uvw} N^{pu} N^{qv} M^{rw} \eps^{abc}_{ijk} M_{pa} M_{qb} M_{rc}
\end{align}
where the sum on the left hand side includes just those $(ijk)$ and $(prq)$ that are
permutations of $(123)$. There are $3!$ choices for $(ijk)$ and $3!$ choices for
$(pqr)$ and thus the left hand side is easily reduced to $(3!)^2 \det M/\det N$ where
$\det N = 1/\det (N^{-1})$. For the right hand side notice that
\begin{align}
   \eps^{ijk}_{uvw} \eps^{abc}_{ijk} = 3! \eps^{abc}_{uvw}
\end{align}
which leads to
\begin{align}
   \det M = \frac{1}{3!} \det N \eps^{abc}_{uvw} M_{pa} M_{qb} M_{rc} N^{pu} N^{qv} N^{rw}
\end{align}

For our RNC metric we will set $N^{ab} = g^{ab}$ and $M_{ij} = g_{ij}(x)$. Since $g^{ab}$ is
of the form ${\rm diag}(-1,1,1,1)$ we have $\det g = -1$ and thus
\begin{align}
   \det g(x) = - \frac{1}{3!} \eps^{abc}_{ijk}\, g_{pa}(x)\, g_{qb}(x)\, g_{rc}(x)\, g^{ip} g^{jq} g^{kr}
\end{align}

The $\eps^{abc}_{ijk}$ can be constructed in Cadabra by applying the \verb|asym| algorithm
to the upper indices of $\delta^a_i \delta^b_j \delta^c_k$. Note that \verb|asym| will
include the $1/3!$ coeffcient as part of its output.

The following code computes $-\det g$ rather than $\det g$.

Note that Calzetta etal. use an opposite sign for $R_{abcd}$ so when comparing the following
results against Calzetta do take note of this flipped sign in $R_{abcd}$.

\clearpage

% ============================================================================================
\section*{The determinant of the metric}

\begin{cadabra}
   {a,b,c,d,e,f,g,h,i,j,k,l,m,n,o,p,q,r,s,t,u#}::Indices(position=independent).

   {a,b,c,d,e,f,g,h,i,j,k,l,m,n,o,p,q,r,s,t,u#}::Integer(1..3).

   \nabla{#}::Derivative.

   d{#}::KroneckerDelta.

   g^{a b}::Symmetric.
   g_{a b}::Symmetric.

   R_{a b c d}::RiemannTensor.

   x^{a}::Weight(label=num,value=1).

   def truncate (obj,n):

       ans = Ex("0")  # create a Cadabra object with value zero

       for i in range (0,n+1):
          foo := @(obj).
          bah  = Ex("num = " + str(i))
          distribute  (foo)
          keep_weight (foo, bah)
          ans = ans + foo

       return ans

   gab := g_{a b}
          - (1/3)   x^{c} x^{d} R_{a c b d}
          - (1/6)   x^{c} x^{d} x^{e} \nabla_{c}{R_{a d b e}}
          + (1/180) x^{c} x^{d} x^{e} x^{f} ( 8 g^{g h} R_{a c d g} R_{b e f h}
                                             -9 \nabla_{c d}{R_{a e b f}} ).          # cdb (ex-10.gab.000,gab)

   iab := g^{a b}
          + (1/3)  x^{c} x^{d} g^{a e} g^{b f} R_{c e d f}
          + (1/6)  x^{c} x^{d} x^{e} g^{a f} g^{b g} \nabla_{c}{R_{d f e g}}
          + (1/60) x^{c} x^{d} x^{e} x^{f} g^{a g} g^{b h}
                                 ( 4 g^{i j} R_{c g d i} R_{e h f j}
                                  +3 \nabla_{c d}{R_{e g f h}} ).                     # cdb(ex-10.iab.000,iab)

   distribute (gab)
   distribute (iab)

   gxab := gx_{a b} -> @(gab).

   eps := d^{a}_{i} d^{b}_{j} d^{c}_{k}.   # cdb (ex-10.eps.001,eps) # includes a factor of 1/3!
   asym (eps,$^{a},^{b},^{c}$)             # cdb (ex-10.eps.002,eps)

   # compute negative detg rather than det g, note 1/3! included in eps
   Ndetg := @(eps) gx_{p a} gx_{q b} gx_{r c} g^{i p} g^{j q} g^{k r}.   # cdb (ex-10.Ndetg.001,Ndetg)

   substitute       (Ndetg,gxab)                                         # cdb (ex-10.Ndetg.002,Ndetg)
   distribute       (Ndetg)                                              # cdb (ex-10.Ndetg.003,Ndetg)
   Ndetg = truncate (Ndetg,4)                                            # cdb (ex-10.Ndetg.004,Ndetg)
   substitute       (Ndetg,$g^{a b} g_{b c} -> d^{a}_{c}$,repeat=True)   # cdb (ex-10.Ndetg.005,Ndetg)
   eliminate_kronecker (Ndetg)                                           # cdb (ex-10.Ndetg.006,Ndetg)
   sort_product     (Ndetg)                                              # cdb (ex-10.Ndetg.007,Ndetg)
   rename_dummies   (Ndetg)                                              # cdb (ex-10.Ndetg.008,Ndetg)
   canonicalise     (Ndetg)                                              # cdb (ex-10.Ndetg.009,Ndetg)

   # introduce the Ricci tensor

   substitute (Ndetg,$R_{a b c d} g^{a c}               -> R_{b d}$,repeat=True)               # cdb (ex-10.Ndetg.101,Ndetg)
   substitute (Ndetg,$\nabla_{a}{R_{b c d e}} g^{b d}   -> \nabla_{a}{R_{c e}}$,repeat=True)   # cdb (ex-10.Ndetg.102,Ndetg)
   substitute (Ndetg,$\nabla_{a b}{R_{c d e f}} g^{c e} -> \nabla_{a b}{R_{d f}}$,repeat=True) # cdb (ex-10.Ndetg.103,Ndetg)

   # the following was based on sqrt-Ndetg.tex

   sqrtNdetg := 1/2 + (1/2) @(Ndetg)
               - (1/8) (1/9) R_{a b} R_{c d} x^{a} x^{b} x^{c} x^{d}
               - (1/4) (1/18) R_{a b} \nabla_{c}{R_{d e}} x^{a} x^{b} x^{c} x^{d} x^{e}.
                                                                         # cdb (ex-10.sqrtNdetg.001,sqrtNdetg)

   sort_product   (sqrtNdetg)                                            # cdb (ex-10.sqrtNdetg.002,sqrtNdetg)
   rename_dummies (sqrtNdetg)                                            # cdb (ex-10.sqrtNdetg.003,sqrtNdetg)
   canonicalise   (sqrtNdetg)                                            # cdb (ex-10.sqrtNdetg.004,sqrtNdetg)

   logNdetg := -1 + @(Ndetg)
              - (1/2) (1/9) R_{a b} R_{c d} x^{a} x^{b} x^{c} x^{d}
              - (1/18) R_{a b} \nabla_{c}{R_{d e}} x^{a} x^{b} x^{c} x^{d} x^{e}.
                                                                         # cdb (ex-10.logNdetg.001,logNdetg)

   sort_product   (logNdetg)                                             # cdb (ex-10.logNdetg.002,logNdetg)
   rename_dummies (logNdetg)                                             # cdb (ex-10.logNdetg.003,logNdetg)
   canonicalise   (logNdetg)                                             # cdb (ex-10.logNdetg.004,logNdetg)

   # =================================================================================================
   # the remaining code is just for pretty printing

   def product_sort (obj):
       substitute (obj,$ x^{a}                            -> A000^{a}               $)
       substitute (obj,$ g^{a b}                          -> A001^{a b}             $)
       substitute (obj,$ \nabla_{c}{R_{a b}}              -> A004_{a b c}           $)
       substitute (obj,$ \nabla_{c d}{R_{a b}}            -> A005_{a b c d}         $)
       substitute (obj,$ \nabla_{c d e}{R_{a b}}          -> A006_{a b c d e}       $)
       substitute (obj,$ \nabla_{c d e f}{R_{a b}}        -> A007_{a b c d e f}     $)
       substitute (obj,$ \nabla_{e}{R_{a b c d}}          -> A008_{a b c d e}       $)
       substitute (obj,$ \nabla_{e f}{R_{a b c d}}        -> A009_{a b c d e f}     $)
       substitute (obj,$ \nabla_{e f g}{R_{a b c d}}      -> A010_{a b c d e f g}   $)
       substitute (obj,$ \nabla_{e f g h}{R_{a b c d}}    -> A011_{a b c d e f g h} $)
       substitute (obj,$ R_{a b}                          -> A002_{a b}             $)
       substitute (obj,$ R_{a b c d}                      -> A003_{a b c d}         $)
       sort_product   (obj)
       rename_dummies (obj)
       substitute (obj,$ A000^{a}                 -> x^{a}                          $)
       substitute (obj,$ A001^{a b}               -> g^{a b}                        $)
       substitute (obj,$ A002_{a b}               -> R_{a b}                        $)
       substitute (obj,$ A003_{a b c d}           -> R_{a b c d}                    $)
       substitute (obj,$ A004_{a b c}             -> \nabla_{c}{R_{a b}}            $)
       substitute (obj,$ A005_{a b c d}           -> \nabla_{c d}{R_{a b}}          $)
       substitute (obj,$ A006_{a b c d e}         -> \nabla_{c d e}{R_{a b}}        $)
       substitute (obj,$ A007_{a b c d e f}       -> \nabla_{c d e f}{R_{a b}}      $)
       substitute (obj,$ A008_{a b c d e}         -> \nabla_{e}{R_{a b c d}}        $)
       substitute (obj,$ A009_{a b c d e f}       -> \nabla_{e f}{R_{a b c d}}      $)
       substitute (obj,$ A010_{a b c d e f g}     -> \nabla_{e f g}{R_{a b c d}}    $)
       substitute (obj,$ A011_{a b c d e f g h}   -> \nabla_{e f g h}{R_{a b c d}}  $)

   def get_term (obj,n):

       x^{a}::Weight(label=xnum).

       foo := @(obj).
       bah  = Ex("xnum = " + str(n))
       keep_weight (foo,bah)

       return foo

   def reformat (obj,scale):
       foo  = Ex(str(scale))
       bah := @(foo) @(obj).
       distribute     (bah)
       product_sort   (bah)
       rename_dummies (bah)
       canonicalise   (bah)
       sort_sum       (bah)
       factor_out     (bah,$x^{a?}$)
       ans := @(bah) / @(foo).
       return ans

   def rescale (obj,scale):
       foo  = Ex(str(scale))
       bah := @(foo) @(obj).
       distribute  (bah)
       factor_out  (bah,$x^{a?}$)
       return bah

   # ---------------------------------------------------------------
   # reformat Ndetg

   Rterm0 = get_term (Ndetg,0)       # cdb (ex-10.Rterm0.701,Rterm0)
   Rterm1 = get_term (Ndetg,1)       # cdb (ex-10.Rterm1.701,Rterm1)
   Rterm2 = get_term (Ndetg,2)       # cdb (ex-10.Rterm2.701,Rterm2)
   Rterm3 = get_term (Ndetg,3)       # cdb (ex-10.Rterm3.701,Rterm3)
   Rterm4 = get_term (Ndetg,4)       # cdb (ex-10.Rterm4.701,Rterm4)

   Rterm0 = reformat (Rterm0,  1)    # cdb (ex-10.Rterm0.702,Rterm0)
   Rterm1 = reformat (Rterm1,  1)    # cdb (ex-10.Rterm1.702,Rterm1)
   Rterm2 = reformat (Rterm2,  3)    # cdb (ex-10.Rterm2.702,Rterm2)
   Rterm3 = reformat (Rterm3,  6)    # cdb (ex-10.Rterm3.702,Rterm3)
   Rterm4 = reformat (Rterm4,180)    # cdb (ex-10.Rterm4.702,Rterm4)

   Ndetg := @(Rterm0) + @(Rterm1) + @(Rterm2) + @(Rterm3) + @(Rterm4).  # cdb (ex-10.Ndetg.701,Ndetg)

   # ---------------------------------------------------------------
   # reformat sqrtNdetg

   Rterm0 = get_term (sqrtNdetg,0)   # cdb (ex-10.Rterm0.801,Rterm0)
   Rterm1 = get_term (sqrtNdetg,1)   # cdb (ex-10.Rterm1.801,Rterm1)
   Rterm2 = get_term (sqrtNdetg,2)   # cdb (ex-10.Rterm2.801,Rterm2)
   Rterm3 = get_term (sqrtNdetg,3)   # cdb (ex-10.Rterm3.801,Rterm3)
   Rterm4 = get_term (sqrtNdetg,4)   # cdb (ex-10.Rterm4.801,Rterm4)

   Rterm0 = reformat (Rterm0,  1)    # cdb (ex-10.Rterm0.802,Rterm0)
   Rterm1 = reformat (Rterm1,  1)    # cdb (ex-10.Rterm1.802,Rterm1)
   Rterm2 = reformat (Rterm2,  6)    # cdb (ex-10.Rterm2.802,Rterm2)
   Rterm3 = reformat (Rterm3, 12)    # cdb (ex-10.Rterm3.802,Rterm3)
   Rterm4 = reformat (Rterm4,360)    # cdb (ex-10.Rterm4.802,Rterm4)

   sqrtNdetg := @(Rterm0) + @(Rterm1) + @(Rterm2) + @(Rterm3) + @(Rterm4).  # cdb (ex-10.sqrtNdetg.801,sqrtNdetg)

   # ---------------------------------------------------------------
   # reformat logNdetg

   Rterm0 = get_term (logNdetg,0)    # cdb (ex-10.Rterm0.801,Rterm0)
   Rterm1 = get_term (logNdetg,1)    # cdb (ex-10.Rterm1.801,Rterm1)
   Rterm2 = get_term (logNdetg,2)    # cdb (ex-10.Rterm2.801,Rterm2)
   Rterm3 = get_term (logNdetg,3)    # cdb (ex-10.Rterm3.801,Rterm3)
   Rterm4 = get_term (logNdetg,4)    # cdb (ex-10.Rterm4.801,Rterm4)

   Rterm0 = reformat (Rterm0,  1)    # cdb (ex-10.Rterm0.802,Rterm0)
   Rterm1 = reformat (Rterm1,  1)    # cdb (ex-10.Rterm1.802,Rterm1)
   Rterm2 = reformat (Rterm2,  3)    # cdb (ex-10.Rterm2.802,Rterm2)
   Rterm3 = reformat (Rterm3,  6)    # cdb (ex-10.Rterm3.802,Rterm3)
   Rterm4 = reformat (Rterm4,180)    # cdb (ex-10.Rterm4.802,Rterm4)

   logNdetg := @(Rterm0) + @(Rterm1) + @(Rterm2) + @(Rterm3) + @(Rterm4).  # cdb (ex-10.logNdetg.901,logNdetg)

   checkpoint.append (Ndetg)
   checkpoint.append (sqrtNdetg)
   checkpoint.append (logNdetg)

\end{cadabra}

\clearpage

% =================================================================================================
\section*{The metric determinant in Riemann normal coordinates}

\def\Vert{\vrule height 10pt depth 3pt width 0.5pt}
\def\LVert{\Vert\hskip 1.75pt}
\def\RVert{\hskip 1pt\Vert}

\begin{dgroup*}
   \Dmath*{-\det g(x) = \cdb{ex-10.Ndetg.701}+\cdots}
\end{dgroup*}

% =================================================================================================
\section*{The volume element in RNC}

If $-\det g(x)$ is non-negative then we also have
%
\begin{dgroup*}
   \Dmath*{\sqrt{-\det g(x)} = \cdb{ex-10.sqrtNdetg.801}+\cdots}
\end{dgroup*}

% =================================================================================================
\section*{The log of -detg in RNC}
%
\begin{dgroup*}
   \Dmath*{\log\left(-\det g(x)\right) = \cdb{ex-10.logNdetg.901}+\cdots}
\end{dgroup*}

Apart from the signs, this matches exactly the expression given by Calzetta etal. (eq. A14)

\clearpage

% ============================================================================================
% export to json format

\bgroup
\CdbSetup{action=hide}
\begin{cadabra}
   for i in range( len(checkpoint) ):
      cdblib.put ('check{:03d}'.format(i),checkpoint[i],checkpoint_file)
\end{cadabra}
\egroup

\end{document}
