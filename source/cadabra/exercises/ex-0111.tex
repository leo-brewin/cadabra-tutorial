\documentclass[12pt]{cdblatex}
\usepackage{exercises}

\begin{document}

% --------------------------------------------------------------------------------------------
\section*{Exercise 1.11 Cycling free indices -- preferred solution}

\begin{cadabra}
   {a,b,c,d,e,f,u,v,w}::Indices.

   expr := A_{a b c}.                                             # cdb (ex-0111.101,expr)

   rule := T_{a b c} -> @(expr).
   expr := T_{b c a}.                                             # cdb (ex-0111.102,expr)

   substitute (expr, rule)                                        # cdb (ex-0111.103,expr)
\end{cadabra}

\begin{align*}
   &\Cdb{ex-0111.101}\\
   &\Cdb{ex-0111.102}\\
   &\Cdb{ex-0111.103}
\end{align*}

\clearpage

% --------------------------------------------------------------------------------------------
\section*{Exercise 1.11 Cycling free indices -- alternative solution}

This alternative solution uses two rounds of Kroncker deltas. It does the job but is not as
simple as the previous solution.

\begin{cadabra}
   {a,b,c,d,e,f,u,v,w}::Indices.

   \delta{#}::KroneckerDelta.

   expr := A_{a b c}.                                             # cdb (ex-0111.201,expr)

   expr := \delta^{a}_{u} \delta^{b}_{v} \delta^{c}_{w} @(expr).  # cdb (ex-0111.202,expr)

   eliminate_kronecker (expr)                                     # cdb (ex-0111.203,expr)

   expr := \delta^{u}_{b} \delta^{v}_{c} \delta^{w}_{a} @(expr).  # cdb (ex-0111.204,expr)

   eliminate_kronecker (expr)                                     # cdb (ex-0111.205,expr)

\end{cadabra}

\begin{align*}
   &\Cdb{ex-0111.201}\\
   &\Cdb{ex-0111.202}\\
   &\Cdb{ex-0111.203}\\
   &\Cdb{ex-0111.204}\\
   &\Cdb{ex-0111.205}
\end{align*}

\end{document}
