\documentclass[12pt]{cdblatex}
\usepackage{exercises}

\begin{document}

% --------------------------------------------------------------------------------------------
\section*{Exercise 2.4 Combining rules -- a problem}

\begin{cadabra}
   {a,b,c,d,e,f,g,h,i,j,k,l,m,n,o,p,q,r,s,t,u#}::Indices(position=independent).

   \nabla{#}::Derivative.
   \partial{#}::PartialDerivative.

   # rules for covariant derivatives of v

   deriv1 := \nabla_{a}{v^{b}} -> \partial_{a}{v^{b}}
                                + \Gamma^{b}_{d a} v^{d}.

   deriv2 := \nabla_{a}{\nabla_{b}{v^{c}}} -> \partial_{a}{\nabla_{b}{v^{c}}}
                                            + \Gamma^{c}_{d a} \nabla_{b}{v^{d}}
                                            - \Gamma^{d}_{b a} \nabla_{d}{v^{c}}.

   # attempt to combine both rules for second covariant derivative of v

   substitute (deriv2,deriv1)       # cdb (ex-0204.101,deriv2)

\end{cadabra}

\vskip 1cm

Note that the call to \verb|substitute| has made changes to both sides of the rule for
\verb|deriv2|. This is not ideal and a better method is developed in the following exercise.

\begin{align*}
   \Cdb{ex-0204.101}
\end{align*}

\end{document}
