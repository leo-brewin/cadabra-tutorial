\def\Date{10 Aug 2022}

\documentclass[12pt]{cdblatex}
\usepackage{exercises}

\begin{document}

% --------------------------------------------------------------------------------------------
\section*{Exercise 3.1 Some symmetries of Riemann}

\begin{cadabra}
   {a,b,c,d,e,f,g,h,i,j,k,l,m,n,o,p,q,r,s,t,u,v,w#}::Indices(position=independent).

   ;::Symbol;

   \partial{#}::PartialDerivative.

   \Gamma^{a}_{b c}::TableauSymmetry(shape={2}, indices={1,2}).

   Rabcd := R^{a}_{b c d} ->   \partial_{c}{\Gamma^{a}_{b d}}
                             - \partial_{d}{\Gamma^{a}_{b c}}
                             + \Gamma^{e}_{b d} \Gamma^{a}_{c e}
                             - \Gamma^{e}_{b c} \Gamma^{a}_{d e}.       # cdb(Rabcd.000,Rabcd)

   dRabcd := R^{a}_{b c d ; e} -> \partial_{e}{R^{a}_{b c d}}
                                + \Gamma^{a}_{f e} R^{f}_{b c d}
                                - \Gamma^{f}_{b e} R^{a}_{f c d}
                                - \Gamma^{f}_{c e} R^{a}_{b f d}
                                - \Gamma^{f}_{d e} R^{a}_{b c f}.       # cdb(dRabcd.000,dRabcd)

\end{cadabra}

\clearpage

% --------------------------------------------------------------------------------------------
\section*{Exercise 3.1 Antisymmetry on last pair of indices}

\begin{cadabra}
   expr := R^{a}_{b c d} + R^{a}_{b d c}.                               # cdb(ex-0301.101,expr)

   substitute (expr, Rabcd)                                             # cdb(ex-0301.102,expr)
\end{cadabra}

\begin{dgroup*}[spread={3pt}]
   \Dmath*{\cdb{ex-0301.101} = \Cdb*{ex-0301.102}}
\end{dgroup*}

\clearpage

% --------------------------------------------------------------------------------------------
\section*{Exercise 3.1 First Bianchi identity}

\begin{cadabra}
   expr := R^{a}_{b c d} + R^{a}_{d b c} + R^{a}_{c d b}.               # cdb(ex-0301.201,expr)

   substitute   (expr, Rabcd)                                           # cdb(ex-0301.202,expr)
   canonicalise (expr)                                                  # cdb(ex-0301.203,expr)
\end{cadabra}

\begin{dgroup*}[spread={3pt}]
   \Dmath*{\cdb{ex-0301.201} = \Cdb*[\hskip2.5cm\hfill]{ex-0301.202}
                             = \Cdb*{ex-0301.203}}
\end{dgroup*}

\clearpage

% --------------------------------------------------------------------------------------------
\section*{Exercise 3.1 Second Bianchi identity}

\begin{cadabra}
   expr := R^{a}_{b c d ; e} + R^{a}_{b e c ; d} + R^{a}_{b d e ; c}.   # cdb(ex-0301.301,expr)

   substitute     (expr, dRabcd)                                        # cdb(ex-0301.302,expr)
   substitute     (expr,  Rabcd)                                        # cdb(ex-0301.303,expr)
   distribute     (expr)                                                # cdb(ex-0301.304,expr)
   product_rule   (expr)                                                # cdb(ex-0301.305,expr)
   sort_product   (expr)                                                # cdb(ex-0301.306,expr)
   rename_dummies (expr)                                                # cdb(ex-0301.307,expr)
   canonicalise   (expr)                                                # cdb(ex-0301.308,expr)
\end{cadabra}

\begin{dgroup*}[spread={3pt}]
   \Dmath*{\cdb{ex-0301.301} = \Cdb*[\hskip2.0cm\hfill]{ex-0301.302}
                             = \Cdb*{ex-0301.303}
                             = \Cdb*{ex-0301.304}}
\end{dgroup*}

\clearpage

\begin{dgroup*}[spread={3pt}]
   \Dmath*{\cdb{ex-0301.301} = \Cdb*[\hskip4.5cm\hfill]{ex-0301.305}
                             = \Cdb*[\hskip4.5cm\hfill]{ex-0301.306}}
\end{dgroup*}

\clearpage

\begin{dgroup*}[spread={3pt}]
   \Dmath*{\cdb{ex-0301.301} = \Cdb*[\hskip3.5cm\hfill]{ex-0301.307}
                             = \Cdb*{ex-0301.308}}
\end{dgroup*}

\end{document}
