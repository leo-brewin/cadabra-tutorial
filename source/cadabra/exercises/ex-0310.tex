\def\Date{10 Aug 2022}

\documentclass[12pt]{cdblatex}
\usepackage{exercises}

\begin{document}

% --------------------------------------------------------------------------------------------
\section*{Exercise 3.10 Example of {\tt repeat=True} in a substitution}

\begin{cadabra}
   {a,b,c,d,e,f,g,h,i,j,k,l,m,n,o,p,q,r,s,u#}::Indices(position=independent).

   foo := A B + A B A B + A B A B A B + A B A B A B A B .   # cdb(ex-0310.foo.001,foo)
   bah := @(foo).                                           # cdb(ex-0310.bah.001,bah)

   substitute (foo,$A B -> A$)                              # cdb(ex-0310.foo.002,foo)
   substitute (bah,$A B -> A$,repeat=True)                  # cdb(ex-0310.bah.002,bah)
\end{cadabra}

\vskip 1cm

Without {\tt repeat=True} only the first match in a product will be susbstituted.
\begin{align*}
   &\cdb*{ex-0310.foo.001}\\
   &\cdb*{ex-0310.foo.002}
\end{align*}

But with {\tt repeat=True} then all matches in a product will be susbstituted.
\begin{align*}
   &\cdb*{ex-0310.bah.001}\\
   &\cdb*{ex-0310.bah.002}
\end{align*}

\end{document}
