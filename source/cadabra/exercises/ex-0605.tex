\documentclass[12pt]{cdblatex}
\usepackage{exercises}

\begin{document}

% --------------------------------------------------------------------------------------------
\section*{Exercise 6.5 Digging into Cadabra's datastructure}

\begin{cadabra}
   {\theta, \varphi}::Coordinate.
   {a,b,c,d,e,f,g,h#}::Indices(values={\theta, \varphi}, position=independent).

   theta{#}::LaTeXForm{"\theta"}.
   varphi{#}::LaTeXForm{"\varphi"}.

   gab := { g_{\theta \theta}   = r**2,
            g_{\varphi \varphi} = r**2 \sin(\theta)**2 }.   # cdb(ex-0605.100,gab)

   metric := g_{a b} -> g_{a b}.  # a trivial rule :)

   evaluate (metric,gab,rhsonly=True)

   indcs = metric[1][2][1][0]                               # cdb(ex-0605.101,indcs)
   compt = metric[1][2][1][1]                               # cdb(ex-0605.102,compt)

   # cdbBeg(print.0605)
   print ('metric = ' + str(metric.input_form())+'\n')  # reveals Cadabra's internal structure for storing metric

   print ('metric[0] = ' + str(metric[0]))
   print ('metric[1] = ' + str(metric[1])+'\n')

   print ('metric[1][0] = ' + str(metric[1][0]))
   print ('metric[1][1] = ' + str(metric[1][1]))
   print ('metric[1][2] = ' + str(metric[1][2])+'\n')

   print ('metric[1][2][1] = '+ str(metric[1][2][1]))
   print ('metric[1][2][1][0] = '+ str(metric[1][2][1][0]))
   print ('metric[1][2][1][1] = '+ str(metric[1][2][1][1]))
   # cdbEnd(print.0605)
\end{cadabra}

\clearpage

\begin{align*}
   &\Cdb{ex-0605.101}\\[10pt]
   &\Cdb{ex-0605.102}
\end{align*}

\begin{align*}
   g_{\varphi\varphi} &= g_{\Cdb{ex-0605.101}}\\
                      &= \Cdb{ex-0605.102}
\end{align*}

\vskip 1cm

\bgroup
\IfFileExists{ex-0605.cdbcopy}%
{\lstinputlisting[backgroundcolor=\color{white}]{ex-0605.cdbcopy}}%
{Where is {\tt ex-0605.cdbcopy}?}
\egroup

\end{document}
