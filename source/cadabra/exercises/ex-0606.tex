\documentclass[12pt]{cdblatex}
\usepackage{exercises}

\begin{document}

% --------------------------------------------------------------------------------------------
\section*{Exercise 6.6 More digging around in Cadabra's datastructure}

\begin{cadabra}
   {\theta, \varphi}::Coordinate.
   {a,b,c,d,e,f,g,h#}::Indices(values={\theta, \varphi}, position=independent).

   \partial{#}::PartialDerivative.

   g^{a b}::InverseMetric.  # essential when using complete (gab, $g^{a b}$)

   Gamma := \Gamma^{a}_{f g} -> 1/2 g^{a b} (   \partial_{g}{g_{b f}}
                                              + \partial_{f}{g_{b g}}
                                              - \partial_{b}{g_{f g}} ).

   Rabcd := R^{d}_{e f g} ->   \partial_{f}{\Gamma^{d}_{e g}}
                             - \partial_{g}{\Gamma^{d}_{e f}}
                             + \Gamma^{d}_{b f} \Gamma^{b}_{e g}
                             - \Gamma^{d}_{b g} \Gamma^{b}_{e f}.

   Rab := R_{a b} -> R^{c}_{a c b}.

   gab := { g_{\theta \theta}   = r**2,
            g_{\varphi \varphi} = r**2 \sin(\theta)**2 }.   # cdb(ex-0606.101,gab)

   complete   (gab, $g^{a b}$)                              # cdb(ex-0606.102,gab)

   substitute (Rabcd, Gamma)
   substitute (Rab, Rabcd)

   evaluate   (Gamma, gab, rhsonly=True)                    # cdb(ex-0606.103,Gamma)
   evaluate   (Rabcd, gab, rhsonly=True)                    # cdb(ex-0606.104,Rabcd)
   evaluate   (Rab,   gab, rhsonly=True)                    # cdb(ex-0606.105,Rab)

   indcs = Rab[1][2][0][0]                                  # cdb(ex-0606.106,indcs)
   compt = Rab[1][2][0][1]                                  # cdb(ex-0606.107,compt)

   # cdbBeg(print.0606)
   print ('Rab = ' + str(Rab.input_form())+'\n')  # reveals Cadabra's internal structure for storing Rab

   print ('Rab[0] = ' + str(Rab[0]))
   print ('Rab[1] = ' + str(Rab[1])+'\n')

   print ('Rab[1][0] = ' + str(Rab[1][0]))
   print ('Rab[1][1] = ' + str(Rab[1][1]))
   print ('Rab[1][2] = ' + str(Rab[1][2])+'\n')

   print ('Rab[1][2][0] = ' + str(Rab[1][2][0]))
   print ('Rab[1][2][0][0] = ' + str(Rab[1][2][0][0]))
   print ('Rab[1][2][0][1] = ' + str(Rab[1][2][0][1]))
   # cdbEnd(print.0606)
\end{cadabra}

\clearpage

\begin{dgroup*}
   \Dmath*{\Cdb*{ex-0606.101}}
   \Dmath*{\Cdb*{ex-0606.102}}
   \Dmath*{\Cdb*{ex-0606.103}}
   \Dmath*{\Cdb*{ex-0606.104}}
   \Dmath*{\Cdb*{ex-0606.105}}
\end{dgroup*}

\begin{align*}
   R_{\varphi\varphi} &= R_{\Cdb{ex-0606.106}}\\
                      &= \Cdb{ex-0606.107}
\end{align*}

\bgroup
\IfFileExists{ex-0606.cdbcopy}%
{\lstinputlisting[backgroundcolor=\color{white}]{ex-0606.cdbcopy}}%
{Where is {\tt ex-0606.cdbcopy}?}
\egroup

\end{document}
